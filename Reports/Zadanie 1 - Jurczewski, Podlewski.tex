\documentclass{classrep}
\usepackage[utf8]{inputenc}
\frenchspacing

\usepackage{graphicx}
\usepackage[usenames,dvipsnames]{color}
\usepackage[hidelinks]{hyperref}
\usepackage{float}

\usepackage{amsmath, amssymb, mathtools}

\usepackage{fancyhdr, lastpage}
\pagestyle{fancyplain}
\fancyhf{}
\renewcommand{\headrulewidth}{0pt}
\cfoot{\thepage\ / \pageref*{LastPage}}

% csharp cmd
\newcommand{\Csharp}{%
  {\settoheight{\dimen0}{C}C\kern-.05em \resizebox{!}{\dimen0}{\raisebox{\depth}{\# }}}}

\studycycle{Informatyka stosowana, studia dzienne, II st.}
\coursesemester{I}

\coursename{Obliczenia ewolucyjne}
\courseyear{2019/2020}

\courseteacher{dr inż. Łukasz Chomątek}
\coursegroup{środa, 14:00}

\author{%
  \\
  \studentinfo[234067@edu.p.lodz.pl]{Bartosz Jurczewski}{234067}\\
  \studentinfo[234106@edu.p.lodz.pl]{Karol Podlewski}{234106}%
}

\title{Zadanie 1: Optymalizacja funkcji wielu zmiennych}

\begin{document}
\maketitle
\thispagestyle{fancyplain}

\clearpage

\section{Cel}

Celem zadania była optymalizacja funkcji wielu zmiennych na zadanym
przedziale. Dla potrzeby opracowanego rozwiązania przyjęliśmy liczbę zmiennych jako 2.


\section{Opis implementacji}

Zadanie zostało zrealizowane przy użyciu frameworka \textit{.NET Core} w wersji 3.1, języka \textit{\Csharp}, z wykorzystaniem bibliotek: \textit{GeneticSharp}\cite{GeneticSharp}.


\section{Wprowadzenie}

\subsection{Algorytm}
W przygotowanym rozwiązaniu użytkownik może wybrać jeden z dwóch algorytmów. \textbf{Reguła 1/5 sukcesu} dokłada do \textbf{SGA} zmienny krok mutacji.

\begin{enumerate}
    \item \textbf{Simple Genetic Algorithm, SGA} - bazowy algorytm genetyczny 
    \item \textbf{Reguła 1/5 sukcesu} - Jeśli przez k kolejnych generacji więcej niż 1/5 mutacji kończy się sukcesem, należy zwiększyć krok mutacji. W przeciwnym wypadku krok mutacji należy zmniejszyć.
\end{enumerate}

\subsection{Krzyżowanie}

W przygotowanym rozwiązaniu użytkownik może wybrać jedną z dwóch metod krzyżowania:

\begin{enumerate}
    \item \textbf{Uniform Crossover} - każdy bit jest losowo wybierany od jednego z dwóch rodziców z równym prawdopodobieństwem,
    \item \textbf{Three Parent Crossover} - selekcja, która polega na wybraniu trzech losowych rodziców. Po kolei porównywany jest każdy bit pierwszego oraz drugiego rodzica - jeżeli bit jest identyczny, przechodzi on do potomstwa, jeżeli się różni - brany jest bit od trzeciego rodzica.
\end{enumerate}

\subsection{Mutacja}

W przygotowanym rozwiązaniu użytkownik może wybrać jedną z dwóch metod mutacji:

\begin{enumerate}
    \item \textbf{Flip Bit Mutation} - odwrócenie wartości wybranego genu (zamiana 0 na 1, 1 na 0),
    \item \textbf{Reverse Sequence Mutation} - bierzemy sekwencję S ograniczoną przez dwie losowo wybrane pozycje $i$ i $j$, takie że $i < j$, którą odwracamy.
\end{enumerate}

\subsection{Selekcja}

W przygotowanym rozwiązaniu użytkownik może wybrać jedną z dwóch metod selekcji:

\begin{enumerate}
    \item \textbf{Elite Selection} - wybiera najlepsze chromosomy pod względem dopasowania,
    \item \textbf{Roulette Wheel Selection} - metoda koła ruletki, która polega na przypisaniu każdemu osobnikowi prawdopodobieństwa selekcji (im większe przystosowanie, tym większe prawdopodobieństwo), a następnie wylosowaniu chromosomów z całej puli.
\end{enumerate}

\subsection{Funkcje celu}

W przygotowanym rozwiązaniu użytkownik może wybrać jedną z dwóch funkcji celu:

\begin{enumerate}
    \item $y=sin(x_1)cos(x_2),$
    \item$ y=sin(x_1)cos(x_2)x_1 - sin(x_2)cos(x_1) $
\end{enumerate}

\section{Wyniki}

Dla każdego wariantu zostały wykonana seria przynajmniej 10 pomiarów. Z uzyskanych wyników usuwano te, które wpadły w wyjątkowo niekorzystne ekstrema lokalne. Wartość dopasowania jest średnią z tych pomiarów.\\

\subsection{Wpływ mutacji i sposobu krzyżowania}

Argumenty stałe dla tej sekcji badań:
\begin{itemize}
    \item[]
    \begin{itemize}
        \item \textbf{Warunek stopu:} 1000 epok,
        \item \textbf{Funkcja celu:} $y=sin(x_1)cos(x_2),$
        \item \textbf{Wartość minimalna:} -100,
        \item \textbf{Wartość maksymalna:} 100,
        \item \textbf{Wielkość populacji:} 50.
    \end{itemize}
\end{itemize}


\begin{table}[H]
\begin{tabular}{|l|l|l|l|c|}
\hline
           & \multicolumn{1}{c|}{\textbf{Krzyżowanie}} & \multicolumn{1}{c|}{\textbf{Mutacja}} & \multicolumn{1}{c|}{\textbf{Selekcja}} & \textbf{Dopasowanie} \\ \hline
\textbf{1} & Uniform                                   & Flip Bit                              & Elite                                  & 0,831                \\ \hline
\textbf{2} & Uniform                                   & Flip Bit                              & Roulette Wheel                         & 0,862                \\ \hline
\textbf{3} & Uniform                                   & Reverse Sequence                      & Elite                                  & 0,963                \\ \hline
\textbf{4} & Uniform                                   & Reverse Sequence                      & Roulette Wheel                         & 0,872                \\ \hline
\textbf{5} & Three Parent                              & Flip Bit                              & Elite                                  & 0,981                \\ \hline
\textbf{6} & Three Parent                              & Flip Bit                              & Roulette Wheel                         & 0,813                \\ \hline
\textbf{7} & Three Parent                              & Reverse Sequence                      & Elite                                  & 0,971                \\ \hline
\textbf{8} & Three Parent                              & Reverse Sequence                      & Roulette Wheel                         & 0,825                \\ \hline
\end{tabular}
\caption{Porównanie wpływu krzyżowania, mutacji oraz selekcji dla algorytmu SGA}
\label{tab:calc1}
\end{table}


\begin{table}[H]
\begin{tabular}{|l|l|l|l|c|}
\hline
           & \multicolumn{1}{c|}{\textbf{Krzyżowanie}} & \multicolumn{1}{c|}{\textbf{Mutacja}} & \multicolumn{1}{c|}{\textbf{Selekcja}} & \textbf{Dopasowanie} \\ \hline
\textbf{9} & Uniform                                   & Flip Bit                              & Elite                                  & 0,844                \\ \hline
\textbf{10} & Uniform                                   & Flip Bit                              & Roulette Wheel                         & 0,827                \\ \hline
\textbf{11} & Uniform                                   & Reverse Sequence                      & Elite                                  & 0,968                \\ \hline
\textbf{12} & Uniform                                   & Reverse Sequence                      & Roulette Wheel                         & 0,951                \\ \hline
\textbf{13} & Three Parent                              & Flip Bit                              & Elite                                  & 0,983                \\ \hline
\textbf{14} & Three Parent                              & Flip Bit                              & Roulette Wheel                         & 0,884                \\ \hline
\textbf{15} & Three Parent                              & Reverse Sequence                      & Elite                                  & 0,991                \\ \hline
\textbf{16} & Three Parent                              & Reverse Sequence                      & Roulette Wheel                         & 0,899                \\ \hline
\end{tabular}
\caption{Porównanie wpływu krzyżowania, mutacji oraz selekcji dla algorytmu One-fifth-rule}
\label{tab:calc2}
\end{table}


\subsubsection*{Dyskusja}
Użycie krzyżowania \textit{Uniform} negatywnie wpłynęło na dopasowanie - niezależnie od algorytmu czy mutacji. Podobnie miała się sprawa z selekcją ruletki, która poza słabszymi wynikami, miała wyjątkową tendencję do wpadania w niekorzystne ekstrema lokalne. Wariantem o najwyższym (czyli najlepszym) wyniku dopasowania dla algorytmu \textit{One-fifth-tule} okazał się wariant z krzyżowaniem \textit{Three Parent}, mutacją \textit{Reverse Sequence}, oraz selekcją \textit{Elite}. Dla algorytmu \textit{SGA} najlepszym wariantem okazał się ten z krzyżowaniem \textit{Three Parent}, mutacją \textit{Flip Bit} oraz selekcją \textit{Elite}. To właśnie te parametry wykorzystamy w dalszej części badań.

\subsection{Porównanie działania dla różnych funkcji celu}

Argumenty stałe dla tej sekcji badań:
\begin{itemize}
    \item[]
    \begin{itemize}
        \item \textbf{Krzyżowanie:} Three Parent Crossover,
        \item \textbf{Selekcja:} Elite Selection,
        \item \textbf{Warunek stopu:} 1000 epok,
        \item \textbf{Wartość minimalna:} -100,
        \item \textbf{Wartość maksymalna:} 100,
        \item \textbf{Wielkość populacji:} 50.
    \end{itemize}
\end{itemize}

\begin{table}[H]
\centering
\begin{tabular}{|l|l|l|c|}
\hline
             & \textbf{Algorytm} & \textbf{Mutacja}  & \textbf{Dopasowanie} \\ \hline
\textbf{17} & SGA                & Flip Bit          & 0,981                \\ \hline
\textbf{18} & One-fifth-rule     & Reverse Sequence  & 0,991                \\ \hline
\end{tabular}
\caption{Porównanie wyników dla pierwszej funkcji celu}
\label{tab:calc3}
\end{table}

\begin{table}[H]
\centering
\begin{tabular}{|l|l|l|c|}
\hline
             & \textbf{Algorytm} & \textbf{Mutacja}  & \textbf{Dopasowanie} \\ \hline
\textbf{19} & SGA               & Flip Bit          & 82,946               \\ \hline
\textbf{20} & One-fifth-rule    &  Reverse Sequence & 84,344               \\ \hline
\end{tabular}
\caption{Porównanie wyników dla drugiej funkcji celu}
\label{tab:calc4}
\end{table}

\subsubsection*{Dyskusja}

Dla obu porównywanych przez nas funkcji celu algorytm \textbf{One-fifth-rule} wykazał lepsze średnie dopasowanie od SGA. To właśnie ten algorytm wykorzystamy w ostatniej części badań.


\subsection{Porównanie działania w zależności od jego parametrów}

Argumenty stałe dla tej sekcji badań:
\begin{itemize}
    \item[]
    \begin{itemize}
        \item \textbf{Algorytm:} One-fifth-rule,
        \item \textbf{Krzyżowanie:} Three Parent Crossover,
        \item \textbf{Mutacja:} Reverse Sequence Mutation,
        \item \textbf{Selekcja:} Elite Selection,
        \item \textbf{Funkcja celu:} $y=sin(x_1)cos(x_2),$
        \item \textbf{Wartość minimalna:} -100,
        \item \textbf{Wartość maksymalna:} 100.
    \end{itemize}
\end{itemize}

\begin{table}[H]
\centering
\begin{tabular}{|l|l|l|c|}
\hline
            & \multicolumn{1}{c|}{\textbf{Populacja}} & \multicolumn{1}{c|}{\textbf{Liczba epok}} & \textbf{Dopasowanie} \\ \hline
\textbf{21} & 25                                      & 500                                       & 0,925                \\ \hline
\textbf{22} & 25                                      & 1000                                      & 0,927                \\ \hline
\textbf{23} & 25                                      & 2000                                      & 0,98                 \\ \hline
\textbf{23} & 50                                      & 500                                       & 0,978                \\ \hline
\textbf{25} & 50                                      & 1000                                      & 0,991                \\ \hline
\textbf{26} & 50                                      & 2000                                      & 0,985                \\ \hline
\textbf{27} & 100                                     & 500                                       & 0,982                \\ \hline
\textbf{28} & 100                                     & 1000                                      & 0,99                 \\ \hline
\textbf{29} & 100                                     & 2000                                      & 0,992                \\ \hline
\end{tabular}
\caption{Porównanie dopasowania w zależności od populacji oraz liczby epok}
\label{tab:calc5}
\end{table}

\subsubsection*{Dyskusja}

Zmiana populacji oraz liczby epok w znaczącym stopniu nie wpływa na wynik dopasowania - jedynie zmniejszenie populacji wpływa na zauważalne pogorszenie wyników. Zwiększenie liczby epok nie koniecznie musi oznaczać lepsze dopasowanie.


\section{Wnioski}

\begin{itemize}
    \item Bardzo ważne jest dopranie odpowiedniego sposobu krzyżowania, mutacji oraz selekcji, przez co możemy poprawić wyniki algorytmu.
    \item Dodatkowe modyfikacje algorytmu w czasie jego działania, takie jak reguła 1/5 sukcesu, pozytywnie wpływają na algorytm optymalizacyjny.
    \item Nie ma potrzeby dopierania nieskończenie dużej wielkości populacji czy liczby epok potrzebnych do ukończenia procesu optymalizacyjnego, gdyż nie wpłynie to znacząco na wyniki, a może wydłużyć czas działania algorytmu.
\end{itemize}

\begin{thebibliography}{0}
    \bibitem{GeneticSharp}
    \textsl{GeneticSharp, \url{https://github.com/giacomelli/GeneticSharp}}
    
    \bibitem{FunctionOptimazation}
    \textsl{Function optimization with GeneticSharp, \url{http://diegogiacomelli.com.br/function-optimization-with-geneticsharp/}}

\end{thebibliography}


\end{document}
